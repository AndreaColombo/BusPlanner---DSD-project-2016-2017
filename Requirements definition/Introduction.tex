\section{INTRODUCTION}
\subsection{Purpose of this document}
The purpose of this document is to specify the functional and nonfunctional requirements of the project. 
\subsection{Document organization}
The document is organized as follows:
\begin{itemize}
	\item Section 1, Introduction section describes the content of this document.
	\item Section 2, Functional requirements section describes the functional requirements of the project as Use cases, Use case descriptions with activity diagrams, and sequence diagrams.
	\item Section 3, Nonfunctional requirements section describes nonfunctional requirements such as availability, security, privacy, data redundancy and performances.
\end{itemize}
\subsection{Intended audience}
The intended audience of this document is: 
\begin{itemize}
	\item Development team, as a guidance during the development activities and for the team to ensure they understand the requirements of the project.
	\item The supervisors who can use this document to understand the future process of the project.
	\item The customer who can ensure that all the requirements are captured by the team.
\end{itemize}
\subsection{Scope}
This document provides the high level requirements description of the project. Both the functional and nonfunctional requirements of the projects are presented using some UML diagrams such as use case diagrams, sequence diagrams and activity diagrams.
\subsection{Definitions and acronyms}
\subsubsection{Definitions}
\begin{center}
	\begin{tabular} { | m{3cm} | m{10cm} | }
		\hline
		\textbf{Keyword} & \textbf{Definitions}\\
		\hline
		User & A person who requests for bus by being from bus stop.\\
		\hline
		Fleet Manager & Who owns the buses. He/she wants to know the utilization of buses and scheduling of buses.\\
		\hline
		User Request & Information generated with timestamps for the scheduling purpose.\\
		\hline
		Algorithm & A method used to enhance the scheduling process which is static as well as dynamic.\\
		\hline
		Sprint & A repeatable work cycle which is also known as iteration.\\
		\hline
	\end{tabular}
\end{center}
\subsubsection{Acronyms and abbreviations}
\begin{center}
	\begin{tabular} { | m{5cm} | m{8cm} | }
		\hline
		\textbf{Acronym/abbreviation} & \textbf{Definitions}\\
		\hline
		UI & User Interface\\
		\hline
		GUI & Graphical User Interface\\
		\hline
		MDH & M\"{a}lardalens H\"{o}gskola, V\"{a}ster\r{a}s, Sweden\\
		\hline
		POLIMI & Politecnico di Milano, Milan, Italy\\
		\hline
		QA & Quality Assurance\\
		\hline
		DSD & Distributed Software Development\\
		\hline
	\end{tabular}
\end{center}