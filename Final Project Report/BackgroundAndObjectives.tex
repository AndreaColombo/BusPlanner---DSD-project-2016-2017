\section{BACKGROUND AND OBJECTIVES}

\subsection{Background}
Johannesburg has a complex bus system which is quite different from what is used elsewhere. There exist a gap of effectiveness and perfect schedule time. Technology has not yet fully revolutionized the process of scheduling as most of the processes are done by manual work. 

These are the main factors that have created a problem when users wanted to schedule their everyday trips. A lot of waiting time, missed and late buses seem to be a normal day routine. Still, finding the perfect scheduling algorithm that will ease the transportation is yet a big step to be achieved. 

A lot of factors, such as the number of passengers that want to use a bus, the alternative routes, etc, have been considered in this process. 

\subsection{Project goal}
In order to solve the above mentioned problems, we implemented an algorithm that can help in the bus planning process. It increases the process' efficiency and reduces the time needed to do a scheduling. The users waiting time should be dropped from hours to minutes.

\subsection{High level description of the functionalities}
The BusPlanner project is based on an algorithm that simulates user requests around the city of Johannesburg. These requests are identified by the two bus stops where the user wants to get on and off the bus. Based on this information, the algorithm is able to identify which route reaches both the user's starting and end point, and then assign the user to the bus already covering that route or assign a new bus to that route if the one already covering it is full.

In a real world users will interact with the system by sending requests for a bus, related to a specific position. They will also be able to view the buses' location in the city, thanks to the mapping service the system makes use of.

On the other hand, bus drivers can see all the user requests, along with all the related information. 

Finally, fleet managers are the company's resource managers: they manage buses, drivers and routes and they have access to all the information related to the past rides.