\section{BACKGROUND AND OBJECTIVES}
\subsection{Overview} 
The purpose of this project is to help the city of Johannesburg with the bus planning process which, as of now, lacks of efficiency and effectiveness. 

For more information: \url{http://www.fer.unizg.hr/_download/repository/Project_Plan\%5B12\%5D.pdf}

\subsection{High level description of the functionalities}
The BusPlanner project is based on an algorithm that simulates user requests around the city of Johannesburg. These requests are identified by the two bus stops where the user wants to get on and off the bus. Based on this information, the algorithm is able to identify which route reaches both the user's starting and end point, and then assign the user to the bus already covering that route or assign a new bus to that route if the one already covering it is full. 

In this way the process efficiency will be improved and the time needed to do a scheduling will be reduced. Also, the users' waiting time will be dropped from hours to minutes. 

In a real world users will interact with the system by sending requests for a bus, related to a specific position. They will also be able to view the buses' location in the city, thanks to the mapping service the system will make use of. 

On the other hand, bus drivers will receive a notification for each user request, along with all the related information. 

Fleet managers are the company's resource managers: they manage buses, drivers and routes and they have access to all the information related to the past rides.