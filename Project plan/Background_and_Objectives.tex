\section{BACKGROUND AND OBJECTIVES}
\subsection{Background}
Johannesburg has a complex bus system which is quite different from what is used elsewhere. There exist a gap of effectiveness and perfect schedule time. Technology has not yet fully revolutionized the process of scheduling as most of the processes are done by manual work. These are the main factors that have created a problem when users wanted to schedule their everyday trips.

A lot of waiting time, missed and late buses seem to be a normal day routine. Still, finding the perfect scheduling algorithm that will ease the transportation is yet a big step to be achieved. A lot of factors, such as the number of passengers that want to use a bus, the alternative routes, etc, are to be considered in this process. 
\subsection{Project goal}
In order to solve the above mentioned problems, it was requested the creation of an algorithm that will help in the bus planning process. It will reduce the number of buses per route and the time needed to do a scheduling. The user’s waiting time should be dropped from hours to minutes. 
\subsection{Requirements}
\begin{itemize}
	\item The fleet manager should map the user requests with the available buses.
	\item The system should be mobile responsive.
	\item The users can generate the requests from a bus stop using the mobile application.
	\item The users can visualize the buses’ location in the city. 
	\item The users’ waiting time must be minimum.
	\item Bus drivers should be notified for upcoming and pending user requests.
	\item Bus drivers can visualize where the requests come from. 
	\item The buses are of two different sizes, each with a maximum number of persons.
	\item The buses should not leave the depot without passengers.
\end{itemize}


